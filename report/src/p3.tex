\problem

\begin{equation}
	\label{eq:p3 plant 1}
	\mathcal{G}_1 = \frac{1}{s^2 + 50 s + 184}
\end{equation}

\begin{equation}
	\label{eq:p3 plant 2}
	\mathcal{G}_2 = \frac{s+16}{s^2 + 50 s + 184}
\end{equation}

\smallskip
\large\textbf{Predicted values}\\
Evaluating the plant transfer functions at $s=0$ gives dc gains of $\mathcal{G}_{1,0}=\num{5.43e-3}$ and $\mathcal{G}_{2,0}=\num{86.96e-3}$.

The formula $t_s\simeq 4\tau=\frac{4}{\sigma}$ gives approximate 2\% settling times for the plants|assuming they can be approximated as first-order systems with a pole at $-\sigma$|of $t_{s,1}=t_{s,2}=\SI{1}{\second}$.

As the plants are being assumed to be first order systems (i.e. they have a dominant pole), there should be 0\% overshoot for either.

\smallskip
\large\textbf{\matlab calculations}\\
\Cref{fig:p3 step} gives the values of the steady-state response, the settling time, and the percentage overshoot for both plants.

The steady state values for booth plants are identical to the predicted values to three significant digits.

The 2\% settling time of the first plant is identical to the predicted value. The settling time of the second plant differs by $0.071$ ($7.1\%$ error) which is quite close.

Neither plant exhibits any overshoot.\\

\ffigures{p3-step.png/Step responses of \cref{eq:p3 plant 1,eq:p3 plant 2}/fig:p3 step}

Using the \texttt{residue} command, the poles of both systems are at $-4$ and $-46$. This gives a partial fraction expansion for both systems (with $A$ and $B$ differing) of 
$$
G(s) = \frac{A}{s+4}+\frac{B}{s+46}
$$
Via the inverse laplace transform, we get the time domain representation:
$$
g(t) = A\exp{-4t}+B\exp{-46t}
$$
Filling in values for $A$ and $B$ for each plant:
\begin{align*}
	g_1(t) &\simeq 0.0238\exp{-4t} - 0.0238\exp{-46t}\\
	g_2(t) &\simeq 0.2857\exp{-4t} + 0.7143\exp{-46t}
\end{align*}
The term with the time constant $\tau=\frac{1}{46}$ will decay much faster than the term with $\tau=\frac{1}{4}$, thus both plants will exhibit behaviour very similar to a first order system i.e. no overshoot.\\

The step response of both plants is given by the product of their transfer function with the laplace transform of the step response, $\frac{1}{s}$. The inverse laplace transform of the equation formed from the partial fraction expansion of the result is the time-domain step response.
\begin{align*}
	\mathcal{G}_1 &= \frac{0.0005}{s+46}- \frac{0.0060}{s+4}+\frac{0.0054}{s}\\
	\mathcal{G}_2 &= -\frac{0.0155}{s+46}- \frac{0.0714}{s+4}+\frac{0.0870}{s}\\
&\text{Applying inverse Laplace}\\
g_1&= 0.0054-{0.0060}\exp{-4t}+{0.0005}\exp{-46t}\\
g_2&= 0.0870- {0.0714}\exp{-4t}-{0.0155}\exp{-46t}
\end{align*}