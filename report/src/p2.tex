\problem

The partial fraction expansion of the plant (given in \cref{eq:p1 plant tf}) when the unit step is the input, as obtained using the \texttt{residue} command, is as follows:
\begin{align*}
	\mathcal{H}(s) = \frac{s+3.6}{s^2+20s+64}\cdot\frac{1}{s} &= -\frac{0.0646}{s+16} + \frac{0.0083}{s+4} + \frac{0.0563}{s}
\end{align*}
Taking the inverse Laplace transform of the above, we arrive at an expression for the step response of the system.
\begin{align*}
	h(t) &= 0.0563 + 0.0083\exp{-4t} -0.0646\exp{-16t}
\end{align*}
% Based on partial fractions explain why the step response shows overshoot.

\textbf{The step response shows overshoot} because the coefficient of the most dominant pole, $s=-4$, has a positive sign. The sum of this and the DC gain will therefore be greater than the steady-state value for some period before the $s=-4$ term approaches zero.

% Explain also why the dominance by the dominant pole is weak. 

\textbf{The dominance by the dominated pole is weak} because the magnitude of the dominant pole's coefficient is a much less than that of the next most significant pole, at $s=-16$ ($\approx8\times$ smaller).

% Estimate the time which must elapse for the component of the transient due to the dominant term to be 20 times greater than the rest of the transient.

The time $t_{d}$ for the dominant term's magnitude to be at least 20 times that of the rest of the transient is calculated as follows:
\begin{align*}
	\frac{0.0083\exp{-4t_d}}{0.0646\exp{-16t_d}} &= 20\\
	-4t_d + 16 t_d&= \ln{\left(\frac{0.0646}{0.0083}\cdot 20\right)}\\
	t_d = \SI{0.4206}{\second}
\end{align*}

% Show that this “time to dominance” is large relative to the expected 2% settling time as predicted by dominant pole theory.

\textbf{Dominant pole theory predicts} a settling time of $t_{s,2\%}=\SI{1}{\second}$. 
$$
\frac{t_d}{t_{s}} = 42\%
$$

42\% is a significant fraction of the settling time.  