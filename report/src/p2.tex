\problem

The partial fraction expansion of the plant (given in \cref{eq:p1 plant tf}) when the unit step is the input, as obtained using the \texttt{residue} command, is as follows:
\begin{align*}
	\mathcal{H}(s) = \frac{s+3.6}{s^2+20s+64}\cdot\frac{1}{s} &= -\frac{0.0646}{s+16} + \frac{0.0083}{s+4} + \frac{0.0563}{s}
\end{align*}
Taking the inverse Laplace transform of the above, we arrive at an expression for the step response of the system.
\begin{align*}
	h(t) &= 0.0563 + 0.0083\exp{-4t} -0.0646\exp{-16t}
\end{align*}
% Based on partial fractions explain why the step response shows overshoot.
% TODO

% Explain also why the dominance by the dominant pole is weak. 
% TODO: How is this different from in P1?

% Estimate the time which must elapse for the component of the transient due to the dominant term to be 20 times greater than the rest of the transient.

% Show that this “time to dominance” is large relative to the expected 2% settling time as predicted by dominant pole theory.